%----------------------------------------------------------------------------
% Abstract in hungarian
%----------------------------------------------------------------------------
\phantomsection\addcontentsline{toc}{chapter}{Kivonat}
\begin{center}
	\huge
	\textbf{Kivonat}
\end{center}


	Dolgozatomban bemutatom a poros plazma kísérletek apparátusát.
	A kísérlet során a kristályrácsba rendeződő részecskékről egy nagy sebességű kamerával fényképek készülnek.
	Ezen képek feldolgozásával a részecskék pozícióját meghatározom és a rendszer statisztikai eloszlásait kiszámítom.
	
	Ismertetem a részecske detektálásának módszerét szűrés és adaptív döntési küszöb használatával.
	Az elterjedt és a háttérzaj kiszűrésére kézenfekvőnek tűnő FIR Gauss szűrő helyett a hatékonyabb medián szűrőt javasolom és alkalmaztam.
	A pozíció számítására a momentum módszert implementáltam, ami nagyobb számítási kapacitást igényel, de szubpixeles felbontást tesz lehetővé.
	
	Ezután áttekintem az OpenCL keretrendszert, amit a párhuzamos program megírásának segítségére használtam.
	Az itt ismertetett megállapításokat és a használandó eszközök tulajdonságait figyelembe véve állítottam össze a párhuzamos program lépéseit, amelyeket részletesen bemutatok.
	Az elkészült programokat CPU-n, GPU-n és multiprocesszoros kártyán is futtatva a futási idejüket összevetettem.
	A programot beillesztem a nagy sebességű kamera képfelvevő szoftverébe az online adatfeldolgozás végett.

\newpage


\phantomsection\addcontentsline{toc}{chapter}{Abstract}
\begin{center}
	\huge
	\textbf{Abstract}
\end{center}

	In my thesis I show the apparatus of the dusty plasma experiment.
	During the experiment the particles form a crystalline structure
        and image sequences are taken of them with high speed camera.
	Processing these images results in the particle's positions
        and statistical distributions of the system.
	
	I describe the detection procedure of the particles using
        filtering and adaptive decision level determination.
	Instead of using the common and trivial FIR Gauss filter I implement the more efficient median filter.
	For computing the particle's position I implement the momentum
        method, which is computationally more demending but it provides sub-pixel resolution.
	
	I give an overview of the OpenCL framework, which is used for parallel programming.
	Applying the above stated framework and taking into account
        the properties of the devices available I composed the
        program's principle steps.
	These steps are optimize to achieve fast run times.
	I benchmark the program's run time on CPU, on GPU and on many integrated core card.
	Finally I combine the program with the image acquiring
        softver of the high speed camera for online image processing..
	
