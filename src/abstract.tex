%----------------------------------------------------------------------------
% Abstract in hungarian
%----------------------------------------------------------------------------
\phantomsection\addcontentsline{toc}{chapter}{Kivonat}
\begin{center}
	\huge
	\textbf{Kivonat}
\end{center}


	Dolgozatomban bemutatom a poros plazma kísérletek apparátusát.
	A kísérlet során a kristályrácsba rendeződő részecskékről egy nagy sebességű kamerával fényképek készülnek.
	Ezen képek feldolgozásával a részecskék pozícióját és a rendszer statisztikai mennyiségeit számítom.
	
	Ismertettem a részecske detektálásának módszerét szűrés és adaptív döntési küszöb használatával.
	Az elterjedt FIR Gauss szűrő helyett a hatékonyabb medián szűrőt javasoltam és alkalmaztam.
	A pozíció számítására a momentum módszert implementáltam, ami nagyobb számítási energiát igényel, de szubpixeles felbontást lehet vele elérni.
	
	Ezután áttekintem az OpenCL keretrendszert, amit a párhuzamos program megírásának segítségére használtam.
	Az itt ismertetett megállapításokat és a használandó eszközök tulajdonságait figyelembe véve állítottam össze a párhuzamos program lépéseit, amit részleteztem is.
	Az elkészült programokat CPU-n, GPU-n és multiprocesszoros kártyán is futtatva a futási idejüket összevetettem.
	A programot beillesztem a nagy sebességű kamera képfelvevő szoftverébe az online feldolgozás végett.

\newpage


\phantomsection\addcontentsline{toc}{chapter}{Abstract}
\begin{center}
	\huge
	\textbf{Abstract}
\end{center}

	In my thesis I show the apparatus of the dusty plasma's experiment.
	During the experiment the particles are in crystal structure and photos are taken of them with high speed camera.
	Processing these images will result the particle's position and the system's statistical quantities.
	
	Than I describe the detection of the particles using filtering and adaptive decision level.
	Instead of using the common FIR Gauss filter I implement the more effective median filter.
	For computing the particle's position I implement the momentum method, which take more steps but one can achieve sub-pixel resolution.
	
	After I overview the OpenCL framework, which is used for parallel programming.
	Using this statements and the available device's properties I composed the program's steps.
	These steps are investigated for achieving faster run time.
	I benchmark the program's run time on CPU, on GPU and on many integrated core card.
	Finally I put the program inside the image acquiring softver of the high speed camera for online processing..
	