% ----------------------------------------------------------------------------
\chapter{A host program bemutatása}
% ----------------------------------------------------------------------------
	A $~100\ \mathrm{FPS}$-el érkező képeket az eszköz globális memóriájának méretét figyelembe véve dolgozom fel.
	Ha az eszköz memóriájába $100$ képnél kevesebb fér be, akkor a másodperc fennmaradó képei eldobásra kerül 
	A megjelenítésnek nem fontos szigorúan real-time működésűnek lennie (soft real-time), adott fokú késleltetés megengedhető 
	(határidő elmulasztása nem jár súlyos következménynel). A program ciklikusan a következő felsorolásban olvasható lépéseket hajtja
	végre. A lépések a későbbi \ref{sec:parallel} részben ismertetettek végett párhuzamosan időben átlapolódva történnek. 
	\begin{enumerate*}
		\item Eszközön futtatandó kernelek inicializálása, argumentumainak beállítása,
		\item Kép fogadása (gyűjtése) a kamerától GigE interfészen keresztül (beolvasása a host-memóriába),
		\item Képek leküldése a host-memóriájából az eszköz globális memóriájába,
		\item Kernelek futtatása az eszközön:
		\begin{enumerate*}
			\item A másodperc első képének (medián) szűrése,
			\item Átlag és szórás számítása az eredeti és a szűrt kép különbségén (differenciális kép),
			\item Adaptív detektálási szint előállítása,
			\item Az első és a fennmaradó képeken detektálás.
		\end{enumerate*}
		\item Kernelek futása után az eredmény az eszköz globális memóriájából a host-memóriájába való
		visszatöltése,
		\item Posztprocesszálás és OpenGL megjelenítés.
	\end{enumerate*}
	A kernel megírása során a korábbi \ref{sec:opencl}. fejezetben említetteket figyelembe kell venni.
	Főként a véges lokális és globális memóriát és a work-itemek számát. A kernelek adat-parallel módon lett megírva.
	
\section{A host program párhuzamos felépítése} \label{sec:parallel}
	A korbábban megengedett késleltetésre és a real-time viselkedésre több tényező rossz hatással van, ezek a következők:
	\begin{itemize}[noitemsep]
	  \item A kamera GigE interfészének jittere,
	  \item Operációs rendszer által futtatott további folyamatai,
	  \item Feldolgozó (szűrő és detektáló) algoritmus futási idejének (ET = execution time) változatakozása. 
	\end{itemize}
	A kernelek közül a medián szűrő, ami elrontja a fix futási időt (Fix E. T.) és azt véletlenné teszi.
	A változó futási időről elmondható, hogy a bemeneti kép értékeitől függ. Pontosabban a medián szűrő ablakain belül található
	pixelek rendezettségétől. Hiszen minél rendezettebb, annál gyorsabban található meg a mediánját.
	A medián számításának worst case execeution time-ja (WCET), pontosabban wc. lépésszáma ismert alogoritmuselméletből, ami korlátos
	és megegyezik - a bemenet $N$ számossága esetén - a $O(N \log N)$ \todo{fel lehet javítani\ldots} értékkel.
	
	Adódik, hogy a host programot több konkurens szálra bontva kerüljön implementálásra az ismert Producer-Consumer
	\cite{EWD:EWD329pub} mintát alkalmazva. Ezzel elérhető, hogy a program feldolgozási sebessége ne a kép fogadásának és a kernel
	futási idejének összege legyen, hanem ezek közül a időben rövidebbig.
	Ez viszont szükségessé teszi, hogy az adatkapcsolatot a két szál között várakozási listával legye biztosítva.
	Így a szálak egymásra várásának csökkentését lehet elérni.
	
	A szálak létrehozását és kezelését a \textit{Boost C++ Libraries} \cite{boost} keretrendszer
	megfelelő függvényhívásai oldják meg. A keretrendszer parancssori argumentumkezelést, szálkezelést, szemafort, várakozási
	listát és atomi működésű operátorokkal rendelkező változókat nyújt a programozó számára.
	A szálak közötti adatkapcsolást FIFO típusú (single producer - single consumer) várakozási listával oldom meg.
	
	A \texttt{Main} szálon kerül implementálásra a kamera képeinek fogadása és a várakozási sorba állítása\footnote{Természetesen a
	\texttt{main} szál a program felhasználó általtörténő futtatása által meghívott függvény, így az említettek előtt még inicializáció is történik. Ezt később részletezem. A szálra producer szálként kell tekintenünk a kamera képének fogadása és
	várakozási listába tétele végett.}.
	További consumer és egyben producer szálon a várakozási sorban található kép elővétele majd feldolgozására kerül sor az
	OpenCL kernel által, aminek eredménye a megjelenítő OpenGL bufferébe kerül letárolásra.
	Végül egy consumer szál a bufferben található eredmény OpenGL-es megjelenítést végzi.
	
	Az implementálandó szálak ``szekvencia diagrammja'' a következő \ref{fig:host_seq} ábrán látható.
	Az ábra alapja UML szekvencia diagramm, amit kiegészítettem az OpenCL kernel
	és az OpenGL callback függvényének futásával. Az ábrán három szál láthetó ezek a Main, CProducer, Consumer.
	A szálak részletes ismertetése a következő részben következnek.
	\newpage
	 
%\usepackage{graphics} is needed for \includegraphics
\begin{figure}[H]
\resizebox{\linewidth}{!}{\begin{sequencediagram}
	%\newthread{m}{\makebox[2cm]{Main}}
	\newthread{m}{Main}
	\newinst{prod}{\makebox[2cm]{CProducer}}
	\tikzstyle{inststyle}+=[left color = green, right color = red, rounded corners=3mm]
	\newinst[2]{ocl}{OpenCL}
	
	\tikzstyle{inststyle}+=[left color = white, right color = white, rounded corners=0mm]
	%\newthread{cons}{\makebox[2cm]{Consumer}}
	\newinst{cons}{Consumer}
	\tikzstyle{inststyle}+=[left color = blue, right color = blue, rounded corners=3mm]
	\newinst[1.5]{ogl}{OpenGL}
		\begin{call}{m}{start()}{cons}{join()}
		%\prelevel
			
		\begin{messcall}{cons}{start()}{ogl}
			\begin{call}{m}{start()}{prod}{join()}
			
			\begin{sdblock}{processingLoop}{}
				\mess{m}{input-queue}{prod}
				
				\begin{call}{prod}{enqueueKernel()}{ocl}{ReadBuffer()}
				\end{call}
				
				
				\begin{sdblock}{glutMainLoop()}{}
					\begin{call}{cons}{}{cons}{}
					\end{call}
				\end{sdblock}
				
				\mess{prod}{output-queue}{ogl}
				
				\begin{call}{ogl}{glutSwapBuffers()}{ogl}{}
				\end{call}
			\end{sdblock}
			
			\end{call}
			
		\end{messcall}
		
		\end{call}
	\end{sequencediagram}


}
	\caption{Host program ``szekvencia diagrammja''}
	\label{fig:host_seq}
\end{figure}

	
\section{Main (producer) szál}
	A program felhasználó által való futtatása által jön létre és a program \texttt{main()} függvényét hajtja végre. A szál a
	parancssori argumentumok feldolgozását, az inicializálásért és a kamera adatfolyamának fogadásáért és letárolásáért felel.
	A paraméterek tárolására és konzisztenciájának megörzésére definiáltam a következő osztályt:
\begin{lstlisting}[language=C++]	
class Params {
public:
	...
	bool		only_global;	// csak globalis memoria hasznalata
	...
	uint		file_N;		// file merete def.: 1024
	uint		nh_N;		// median ablakanak merete !!! paratlan !!!
	uint		tail;		// a szures altal letrejovo pixelek a kep szelen 
	uint		Bfile_N;		// az igy kapott kep meret

	uint		pplN;		// kivant local meret

	uint		localN;		// work-group nagysage
	uint		globalN;		// osszes work-item szamossaga

	ulong	aSize;		// eszkoz global memoria max alloc. merete byte-ban
	ulong	lSize;		// eszkoz local memoria meret byte-ban

	ulong	mCuint;		// eszkoz max compute unit szama
};
\end{lstlisting}
	A paraméterek értelmezése a következő:
	\begin{description}[noitemsep]
	\item[only\_global] Csak globális memória haszálata vagy lokálisat is használjon. Adott eszközök esetén a lokális memória a
	globális memóriába van mappelve, így használata csupán felesleges adatmozgatást jelentene $\texttt{only\_global} = 0$,
	\item[file\_N] A kép 2D-s mérete, $\texttt{file\_N} = 1024$,
	\item[nh\_N] A medián szűrő mozgó ablakának 2D-s mérete (páratlan szám) pl.: $\texttt{nh\_N} = 3,5,7,9$ (magasabb fokú szűrőt
	nem érdemes használni, mivel nagy nemlinearitással rendelkezne),
	\item[tail] a szűrés által letrejövő pixelek a kép szélén, azaz \texttt{tail = (nh\_N -1) / 2},
	\item[Bfile\_N] az így kapot kép 2D-s mérete, azaz \texttt{Bfile\_N = file\_N + 2*tail},
	\item[pplN] javasolt lokális méret, pl.: a compute unit-ok száma,
	\item[localN] a tényleges lokális méret egy work-group-on belül,
	\item[globalN] az összes work-item számossága,
	\item[aSize] az eszköz globális memóriájában allokálható maximális memória méret\\
		(\texttt{CL\_DEVICE\_MAX\_MEM\_ALLOC\_SIZE}),
	\item[lSize] az eszköz lokális memóriájának mérete\\
		(\texttt{CL\_DEVICE\_LOCAL\_MEM\_SIZE}),
	\item[mCuint] az eszköz egyszerre futtatható szálának (compute unit) száma \\
		(\texttt{CL\_DEVICE\_MAX\_COMPUTE\_UNITS}).
	\end{description}
	%Értékük a következő részben kerül ismertetésre.\todo{inkább itt.}
	

	\subsection*{Inicializálás}
	A korábban említett várakozási lista fix méretű a $100\ \mathrm{FPS}$-el érkező képek $1$ másodpercnyi feldolgozásához szükséges
	méretű. Ennek megfelelően mivel két várakozási listára van szükség és a kamera egy képe minkettőn ``végigmegy'', így
	\texttt{BUFF\_N=50} hosszúsággal kerül inicializálásra és debug esetén \texttt{Mimage} osztályt illetve release esetén
	\texttt{uint8\_t*} pointereket tartalmaz, amik közvetlenül/közvetve a képre mutatnak.
	A képek $\texttt{BUFF\_N} \times 1024 \times 1024$ \texttt{uint8\_t} típusú tömbben kerül tárolásra.
	Az \texttt{Mimage} osztályt a következőképpen definiáltam:
	
 %float=!H ,caption=Képet tartalmazó osztály]
\begin{lstlisting}[language=C++]
class Mimage {
public:
	unsigned int	i;
	unsigned int	N;
	uint8_t	*ptr;
	
	Oimage();
	Oimage(unsigned int _i, unsigned int _N, unsigned char *_ptr);
};
\end{lstlisting}

\noindent Definiálásra és inicializálásra kerül két single-producer single-consumer várakozási lista, amibe az előbbi osztály
példányai kerülnek. Továbbá minden lista mellett a hozzá tartozó buffer tömb megcímzésére alkalmas atomi művelet végrehajtással
rendelkező változó és a kölcsönös kizárást biztosító szemafor is definiálásra kerül:

\begin{lstlisting}[language=C++]
uint8_t *input = new uint8_t[BUFF_N * pms.Bfile_N*pms.Bfile_N];
uint8_t *output = new uint8_t[BUFF_N * pms.Bfile_N*pms.Bfile_N];

boost::lockfree::spsc_queue<Mimage,boost::lockfree::capacity<BUFF_N>> input_queue;
boost::lockfree::spsc_queue<Mimage,boost::lockfree::capacity<BUFF_N>> output_queue;

boost::mutex input_mtx;
boost::mutex output_mtx;

boost::atomic<int> in_N(-1);
boost::atomic<int> out_N(-1);
\end{lstlisting}

\noindent Az \texttt{input\_queue}-ba a kamera képéhez tartozó, az \texttt{output\_queue}-ba az OpenCL-es feldolgozás utáni
	\texttt{Mimage} osztály példánya kerül. A szemaforok és az atomi változók a producer és a consumer szálak szoftveres/hardveres
	párhuzamos futása végett van szükség, elkerülve a szálak között fennáló versenyhelyzetet.

	Ezután a Consumer és a Producer szál létrehozása (forkja) következik.
	
	Végül a kamera inicializálására kerül sor, ami megfelelően konfigurált hálózati kapcsolat esetén megtalálja a kamera IP és MAC
	címét és ezzel inicializálja a hozzá tartozó, globális változóként deklarált osztályt. A kamera számunka fontos paraméterei ezután
	beállításra kerülnek, mint pédául a felbontás, FPS, expozíció és a küldött IP csomag mérete\footnote{Növelésével az IP csomag
	fejléce okozta overhead és CPU kihasználtság csökkenthető (Jumbo packet).}.
	Ezen beállítások után a kamera stream-hez \texttt{StreamCBFunc} callback függvény kerül hozzárendelésre. 
	 
	\subsection*{Kamera adatfolyamának fogadása}
	Az \texttt{AcquisitionStart} parancs kiadása után megkezdődik a felvétel készítés. Adott frame megérkezésekor a korábbi callback
	függvény kerül meghívásra. Az \texttt{input} bufferbe történő mentése előtt a hozzá tartozó \texttt{input\_mtx}
	szemafor által az erőforrást lefoglalja. A bufferbe történő mentés a hozzá tartozó \texttt{in\_N} atomi változó által
	meghatározott indexű területre történik.
	
\begin{lstlisting}[language=C++]
input_mtx.lock();						// input buffer (eroforras) lefoglalasa

in_N++;
in_N = in_N % BUFF_N;					// cirkularis korbeforogas vegett
 
Mimage iim(in_N, pms.Bfile_N, &input[in_N*pms.Bfile_N*pms.Bfile_N]);

while(!input_queue.write_available()) {;}	// a buffer/varakozasi lista kiurulesere varas

if((*pAqImageInfo).iImageSize != pms.file_N*pms.file_N) {
	std::cout << std::endl << "wrong camera image size" << std::endl;
	exit(EXIT_FAILURE);
}

/*
* frame elmentese in_N-edik helyre
*/
for(uint a = 0; a < pms.file_N; a++) {
for(uint b = 0; b < pms.file_N; b++) {
	input[in_N*pms.Bfile_N*pms.Bfile_N + (a+pms.tail)*pms.Bfile_N + (b+pms.tail)] = (*pAqImageInfo).pImageBuffer[a*pms.file_N + b];
}
}

while(!input_queue.push(iim)) { ; }			// a varakosazi lista berakas

input_mtx.unlock(); // eroforras felszabaditasa (consumer mostmar dolgozhat rajta)
\end{lstlisting}
	
	A fogadott kép az imput bufferbe közepébe kerül bemásolásra. A buffer szélei zérus értékű marad. Erre a kiegészítésre a későbbi
	szűrés során alkalmazott mozgó ablak végett van szükség.
	
\section{CProducer szál}
Ezen szál feldolgozza az \texttt{input\_queue} várakozási listában található képeket az OpenCL kernelek meghívásával és annak
eredményét az \texttt{output\_queue}-ba sorakoztatja fel a későbbi megjelenítés végett.
	
	\subsection{OpenCL inicializálás}
	Az inicializáláshoz első körben szükség van az eszköz fontosabb tulajdonságaira. Az eszköz globális, lokális memóriájának mérete,
	a globális memóriában maximálisan allokálható memória mérete és a compute-unite-ok száma.
	\subsubsection{Globalis memória mérete}
	A globális memóriában a következőknek kell elférnie:
	\begin{itemize}[noitemsep]
	  \item a képek (\texttt{[file\_N][file\_N]}),
	  \item a szűrt képek (\texttt{[Bfile\_N][Bfile\_N]}),
	  \item a detektálás után megjelölt pixelek (\texttt{[file\_N][file\_N]}).
	\end{itemize}
	
	\subsubsection{Lokális memória mérete}
	A lokális memória nagy sebessége és gyors elérése végett alapvető a preferáltsága a párhuzamos applikációkban.
	A program futása során kerül megállíptásra a használt értéke.
	
	A szűrt kép minden pixelének kiszámításához egy work-item-et rendelek, így egy work-item-hez a medián szűrő ablakának megfelelő
	méretű, \texttt{nh\_N $\times$ nh\_N} darab lokális memóriát rendelek. Ezáltal a szűrés teljes mértékben a lokális memóriában
	történik, ezzel lehet optimális OpenCL kódot írni.
	
	\subsection{Kép kernelekkel történő feldolgozása}
	A kernelek fordítása és a megfelelő argumetnumok beállítása után először a medián szűrés, majd a detektálási szint számítása
	végül a detektálás történik. Hibakeresés során a közbülső eredmények is visszaolvasásra kerül.
	
	
\section{Consumer szál}
	Először az OpenGL inicializálás és a megjelenítő ablak létrehozása történik. 
	Ezután a rajzolásra és az időzítésre alkalmas callback függvények kerülnek regisztálásra. Az időzítő függvény az aktuális
	megjelenítés i frekvenciát átlagolással számítja továbbá ennek megfelelően beállítja a következő rajzolás határidejét. A rajzoló
	callback függvény a kimeneti \texttt{output\_queue}-ből kivesz egy képet (elemet), majd azt az OpenGL bufferébe másolja. Továbbá
	a várakozási lista annyi elemét törli ki (dobja el), hogy a korábban mért FPS érték szerint mind megjelíthető legyen.  