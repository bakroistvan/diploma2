% ----------------------------------------------------------------------------
\chapter{Multiprocesszoros program}
% ----------------------------------------------------------------------------
\section{A program lépéseinek bemutatása}
	A $100\ \mathrm{FPS}$-el érkező képeket \textbf{tizesével} dolgozom fel, mivel a megjelenítésnek
	nem fontos real-time működésűnek lennie. 
	A program ciklikusan a következő lépéseket hajtja végre:
	\begin{enumerate*}
		\item Képek beolvasása a host-memóriába,
		\item Képek leküldése a host-memóriájából az eszköz globális memóriájába,
		\item Eszközön futtatandó kernel inicializálása, argumentumainak beállítása,
		\item Kernel futtatása az eszközön,
		\item Kernel futása után az eredmény az eszköz globális memóriájából a host-memóriájába való
		visszatöltése,
		\item Posztprocesszálás,
	\end{enumerate*}
	A kernel megírása során a korábbi \ref{sec:opencl}. fejezetben említetteket figyelembe kell venni.
	Főként a véges lokális és globális memóriát. A kernel adat-parallel módon lett megírva.
	A lokális memória $16 \mathrm{KByte}$ méretű, amit két $8 \mathrm{KByte}$ nagyságú $A$ és $B$
	bufferre osztottam fel.
	
	\noindent A kernel program lépései a következők:
	\begin{enumerate*}
		\item A work-item globális és lokális indexének meghatározása,
		\item Medián szűrés:
		\begin{enumerate*}
			\item A kép egy részének a globális memóriából a lokális $A$ bufferbe való másolása,
			\item Az összes work-item másolási folyamatának megvárása,
			\item Medián szűrés az $A$ bufferből a $B$ bufferba,
			\item Az $A$ bufferba az eredeti ($A$ buffer) és a szűrt ($B$ buffer) különbségének az eredményét
			(differenciális kép) elhelyezni,
			\item Döntési szint számítása és detektálás/megjelölés a $B$ bufferba,
			\item A $B$ bufferben lévő eredmény a globális memóriába való kiírása hibakeresés biztosítása
			végett.
		\end{enumerate*}
		\item Kiterjesztés és a flood-fill algoritmussal a ROI meghatározása:
		\begin{enumerate*}
			\item Megjelölt pixel keresése,
			\item Adott környezetére való kiterjesztése,
			\item A kiterjesztés során a két legtávolabbi pont lesz a ROI határpontjai.
		\end{enumerate*}
		\item Részecske pozíciójának számítása momentum módszerrel,
		\item Eredmény mentése a globális memóriába
	\end{enumerate*}
	
	A kernel lefutása után az eszköz globális memóriájából az eredményeket a hoszt-memóriájába töltjük.
	A számításigényes szűrés, detektálás és momentum számítás az eszközön hajtódott végre. A részecskék
	pozíciójának eloszlásának számítása memóriaigényes, de nem számításigényes feladat, így az a
	host-programban került megvalósításra.

	
% \begin{lstlisting}[frame=single,float=!ht,caption=A detektálás kernelének kódja,
% label=listing:kernel]
% asd;
% \end{lstlisting}