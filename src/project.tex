%--------------------------------------------------------------------------------------
% Feladatkiiras (a tanszeken atveheto, kinyomtatott valtozat)
%--------------------------------------------------------------------------------------
\chapter*{Feladatkiírás}\addcontentsline{toc}{chapter}{Feladatkiírás}

A modern alacsony hőmérsékletű plazmafizikai kísérletek egy új, érdekes és izgalmas területe a poros plazmák
kutatása. Egy elektromos gázkisülésbe helyezett apró (mikrométer méretű) szilárd szemcse a kisülési plazma
atomi részecskéivel kölcsönhatva elektromosan feltöltődik. A sok töltött szemcséből kialakuló elrendezésben a
szilárdtestfizikai jelenségek széles spektruma figyelhető meg, pl. kristályrács kialakulása, fázisátalakulás,
diszlokációk dinamikája, transzport folyamatok, stb. Poros plazmákat jelenleg leginkább alapkutatásokban
alkalmaznak, de jelentőségük az elektronikai gyártásban, fúziós reaktorok üzemeltelésében, Teraherz
technológiában egyre inkább előtérbe kerül.

A kísérleti adatgyűjtés és feldolgozás nagyrésze részecske-követő velocimetrián (particle tracking velocimetry)
alapul, vagyis első lépésben egy nagysebességű kamera segítségével nagyfelbontású képek készülnek, amely
képek segítségével a porszemcsék pontos (a kamera felbontásánál pontosabb) koordinátáit kell meghatározni. A
képek elemzése ezidáig csak a mérést követően, hosszú idő alatt volt megvalósítható a vizsgálandó nagy
adatmennyiség miatt. A multiprocesszoros környezetek segítségével a feldolgozás gyorsítása lehetséges akár
több nagyságrenddel is.

A jelölt feladata, hogy a meglévő kísérleti elrendezés, amely az MTA Wigner Fizikai Kutatóközpont
Szilárdtestfizikai és Optikai Intézezetben található, kiegészítésével a mérés közbeni feldolgozással a mérést
segítő analízist hajtson végre. Ennek eredményével a mérés előkészítése és elvégzése lényegesen gyorsulhatnak.

\begin{flushleft}
\textbf{A jelölt feladata}

\begin{itemize}
	\item Mutassa be a mérési elrendezést és elemezze a kapott adatokat! (Mutassa
	be a mérést!)
	\item Elemezze a lehetséges multiprocesszoros környezeteket, a feladat
	szempontjából lényeges paraméterek és feladatvégrehajtási elvárások szempontjából!
	\item Készítsen programot, amely az azonnali (valós idejű) analízisben résztvevő paramétereket számítja ki, a
	multiprocesszoros környezet kihasználása nélkül!
	\item Készítsen programot, amely a mérési környezetbe illeszkedve a mérésnél
	valós időben képes a vizsgált paraméterek megjelenítésére! Mutassa be és
	elemezze az elkészített programot!
	\item Hasonlítsa össze a multiprocesszoros és a nem-multiprocesszoros
	környezetre elkészített programokat erőforrás igény illetve egyéb paraméterek szempontjából!
\end{itemize}
\end{flushleft}

\begin{flushleft}
\textbf{Irodalom}:
\begin{itemize}
	\item \cite{Hartmann2010} Hartmann P, et. al. ; “Crystallization Dynamics of a Single Layer Complex Plasma”;
	Phys. Rev. Lett., 105 (2010) 115004
	\item \cite{Hartmann2013} Hartmann P, et. al. ; “Magnetoplasmons in Rotating Dusty	Plasmas”, Phys.	Rev. Lett.
	111, 155002 (2013)
	\item \cite{HartmannP2013} Hartmann P, Donkó I, Donkó Z; “Single exposure three-dimensional imaging of dusty
	plasma clusters”; Rev. Sci. Instrum., 84 (2013) 023501/1-5;
\end{itemize}
\end{flushleft}

\begin{flushleft}
\vspace*{1cm}
\textbf{Tanszéki konzulens}: Reichardt András, egy. tanársegéd\\
\textbf{Külső konzulens}: Hartmann Péter, PhD., tud. főmunkatárs (MTA Wigner FK, SZFKI)
\end{flushleft}



\begin{flushleft}
\vspace*{1cm}
Budapest, 2014.03.10.
\end{flushleft}
