%----------------------------------------------------------------------------
\chapter{Összegzés}
%----------------------------------------------------------------------------
	Dolgozatomban bemutattam a porosplazma kísérletek apparátusát. A kísérlet során a kristályrácsba
	rendeződő részecskékről egy nagysebességű kamerával fényképek készülnek. A dolgozatomban ezen
	képeket, kellett feldolgoznom és a részecskék pozícióját detektálnom, amely pozíciók a fizikai modell/szimuláció validálására
	szolgálnak.
	
	Ismertettem a részecske detektálásának módszerét szűrés és adaptív döntési küszöb használatával.
	Az elterjedt FIR Gauss szűrő helyett a hatékonyabb medián szűrőt javasoltam és alkalmaztam. A
	pozíció számítására a momentum módszert implementáltam, ami nagyobb számítási energiát igényel, de
	szubpixeles felbontást tudtam vele elérni. Konstatáltam, hogy az így kialakult program masszívan párhuzamosítható.
	
	Ezután áttekintettem az OpenCL keretrendszert, amit a párhuzamos program megírásának segítségére
	használtam. Az itt ismertetett megállapításokat figyelembe véve állítottam össze a párhuzamos
	program lépéseit, amit részleteztem is. Kiemelném az OpenCL keretrendszer használatának előnyét, ami a platform és párhumazos
	számításra képes eszköztől való függetlenség.
	
	A program első verziója korábban készült és letárolt fájlok feldolgozását végezte és eredményét fájlba mentette el. Míg a végső
	program a kamera képét dolgozta online dolgozza fel és eredményét OpenGL használatával meg is jeleníti.
	
	\section*{További fejlesztési lehetőségek}
	\begin{itemize}
		\item Felhasználó által választott platformon belül taláható összes eszköz számítási teljesítményének kihasználása,
		\item A CProducer szál által futtatási sorba rakott kernel befejezésére ne tétlenül várjon, ezzel javítva a feldolgozási
		sebességen.
	\end{itemize}