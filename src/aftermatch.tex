%----------------------------------------------------------------------------
\chapter{Összegzés}
%----------------------------------------------------------------------------

	Dolgozatomban bemutattam a porosplazma kísérletek apparátusát. A kísérlet során a kristályrácsba
	rendeződő részecskékről egy nagysebességű kamerával fényképek készülnek. A dolgozatomban ezen
	képeket, kellett feldolgoznom és a részecskék pozícióját detektálnom.
	A pozíciók a fizikai modell/szimuláció validálására szolgálnak.
	
	Ismertettem a részecske detektálásának módszerét szűrés és adaptív döntési küszöb használatával.
	Az elterjedt FIR Gauss szűrő helyett a hatékonyabb medián szűrőt javasoltam és alkalmaztam. A
	pozíció számítására a momentum módszert implementáltam, ami nagyobb számítási energiát igényel, de
	szubpixeles felbontást tudtam vele elérni. Konstatáltam, hogy az így kialakult program masszívan párhuzamosítható.
	
	Ezután áttekintettem az OpenCL keretrendszert, amit a párhuzamos program megírásának segítségére
	használtam. Az itt ismertetett megállapításokat figyelembe véve állítottam össze a párhuzamos
	program lépéseit, amit részleteztem is.
	
	Végül az elkészült programot CPU-n és GPU-n is futtatva a futási idejüket összevetettem és
	azonosítottam a gyorsulás forrását kitérve a processzormagra és a memóriájára.
	
	\section*{További feladatok:}
	\begin{itemize}
		\item A host-program real-time mérésbe helyezése egy producer-consumer sémájú szál megoldás
		alkalmazásával,
		\item Az eredmény grafikus  megjelenítése pl.: OpenGL használtatával,
		\item Az OpenCL szabvány által specifikált vektor műveletek támogatásának kiaknázása, ami az Intel
		Xeon PHI processzorkártyában rejlő teljesítményt ki tudná aknázni.
	\end{itemize}
	
	