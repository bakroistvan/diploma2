%----------------------------------------------------------------------------
\chapter{Összegzés}
%----------------------------------------------------------------------------
	Dolgozatomban bemutattam a poros plazma kísérletek apparátusát. A kísérlet során a kristályrácsba
	rendeződő részecskékről egy nagysebességű kamerával felvételek készülnek. A diplomamunkám során ezen
	képeket kellett feldolgoznom és a részecskék pozícióját detektálnom, ami fontos információkat jelent a rendszer paramétereinek
	optimalizálása során.

	Ismertettem a részecske detektálásának módszerét szűrés és adaptív döntési küszöb használatával.
	Az elterjedt FIR Gauss szűrő helyett a hatékonyabb medián szűrőt javasoltam és alkalmaztam. A
	pozíció számítására a momentum módszert implementáltam, ami nagyobb számítási kapacitást igényel, de
	szubpixeles felbontást tudtam vele elérni. Megállapítottam, hogy az így kialakult program masszívan párhuzamosítható.
	
	Ezután áttekintettem az OpenCL keretrendszert, amit a párhuzamos program megírásának segítségére
	használtam. Az itt ismertetett megállapításokat figyelembe véve állítottam össze a párhuzamos
	program lépéseit, amit részleteztem is. Kiemelném az OpenCL keretrendszer használatának előnyét, ami a platform és párhuzamos
	számításra képes eszköztől való függetlenség.
	
	A program jelen változtában alkalmas arra, hogy a kísérletek
        beállítása során értékes, valós idejű visszajelzést adjon a
        kísérletezőnek. Ezáltal az elvégzendő kísérletek hibaszázaléka
        csökkenthető, a kísérleti paraméterek időben optimalizálhatók.
	
	\section*{További fejlesztési lehetőségek}
	\begin{itemize}
		\item Felhasználó által választott platformon belül taláható összes eszköz számítási teljesítményének kihasználása,
		\item A CProducer szál által futtatási sorba rakott kernel befejezésére ne tétlenül várjon, ezzel javítva a feldolgozási
		sebességen.
	\end{itemize}
