% ----------------------------------------------------------------------------
\chapter{A kernel programok lépéseinek bemutatása}
% ----------------------------------------------------------------------------
A lokális memóriát két azonos nagyságú $A$ és $B$ bufferre osztottam fel.

\section{Medián szűrés}
	\noindent A kernel program lépései a következők:
	\begin{enumerate*}
		\item A work-item globális és lokális indexének meghatározása,
		\item Medián szűrés:
		\item A kép egy részének a globális memóriából a lokális $A$ bufferbe való másolása,
		\item Az összes work-item másolási folyamatának megvárása,
		\item Medián szűrés az $A$ bufferből a $B$ bufferba,
		\item Az $A$ bufferba az eredeti ($A$ buffer) és a szűrt ($B$ buffer) különbségének az eredményét
		(differenciális kép) elhelyezni,
		\item Döntési szint számítása és detektálás/megjelölés a $B$ bufferba,
		\item A $B$ bufferben lévő eredmény a globális memóriába való kiírása hibakeresés biztosítása végett.
	\end{enumerate*}
	
	A kernel lefutása után az eszköz globális memóriájából az eredményeket a hoszt-memóriájába töltjük.
	A számításigényes szűrés, detektálás és momentum számítás az eszközön hajtódott végre. A részecskék
	pozíciójának eloszlásának számítása memóriaigényes, de nem számításigényes feladat, így az a
	host-programban került megvalósításra.

\section{Átlagolás}
\dots
	\begin{enumerate*}
		\item A work-item globális és lokális indexének meghatározása,
		\item Eredmény mentése a globális memóriába
	\end{enumerate*}

\section{Detektálás}
\dots
	\begin{enumerate*}
		\item A work-item globális és lokális indexének meghatározása,
		\item Kiterjesztés és a flood-fill algoritmussal a ROI meghatározása:
		\begin{enumerate*}
			\item Megjelölt pixel keresése,
			\item Adott környezetére való kiterjesztése,
			\item A kiterjesztés során a két legtávolabbi pont lesz a ROI határpontjai.
		\end{enumerate*}
		\item Részecske pozíciójának számítása momentum módszerrel,
		\item Eredmény mentése a globális memóriába
	\end{enumerate*}

% \begin{lstlisting}[frame=single,float=!ht,caption=A detektálás kernelének kódja,
% label=listing:kernel]
% asd;
% \end{lstlisting}