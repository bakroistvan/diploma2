% ----------------------------------------------------------------------------
\chapter{A kernel programok lépéseinek bemutatása}
% ----------------------------------------------------------------------------
Az összes work-item azonos kernel programot futtat párhuzamosan, ami következtében a program első lépéseként le kell kérdeznie a
work-item indexeit. Ezt a következő parancsokkal lehetséges megtenni:
\begin{lstlisting}[language=C++]
//------------------------		item-indexes		------------------
// group-ids
size_t ggi	= get_group_id(0);
size_t ggj	= get_group_id(1);
// global-ids
size_t Gi	= get_global_id(0);
size_t Gj	= get_global_id(1);
// local-ids
size_t li	= get_local_id(0);
size_t lj	= get_local_id(1);

//------------------------		dimensions		------------------
size_t lN	= get_local_size(0);
size_t gN	= get_num_groups(0);
\end{lstlisting}

\noindent Ezután a kernelek különböző feladatokat látnak el, amik bemutatása következik. 
\section{Medián szűrés}
	A szűrő feladata a részecskék eliminálása illetve a háttér lehető legalakhűbb átengedése, amjd a differenciális kép létrehozása.
	
\noindent A kernel függvényének fejléce a következő:
\begin{lstlisting}[language=C++]
__kernel void median(	uint		N,		// 0
					uint		file_N,	// 1
					uint		nh_N,	// 2
		__global		uchar	*im,		// 3 :: [N][Bfile_N][Bfile_N]					
		__global		uchar	*Oim,	// 4 :: [N][file_N][file_N]
		__local		uchar	*Lwork	// 5 :: [lN][lN][nh_N][nh_N]
)
\end{lstlisting}

\noindent Az argumentumok értelmezése a következő
	\begin{description}[noitemsep]
		\item[\texttt{N}] az \texttt{im} bufferben tárolt, szűrendő képek száma,
		\item[\texttt{file\_N}] a képek oldalhossza pixelben,
		\item[\texttt{nh\_N}] a szűrő mozgó ablakának mérete,
		\item[\texttt{im}] a szűrendő (bemeneti) képeket tartalmazó globális bufferre mutató pointer,
		\item[\texttt{Oim}] az eredeté és a szűrt kép különbségét tartalmazó globális bufferre (kimenet) mutató pointer,
		\item[\texttt{Lwork}] a szűrés során használt lokális memóriaterületre mutató pointer.
	\end{description}
	
\noindent A kernel program lépései a következők:
	\begin{enumerate*}
		\item A work-item globális és lokális indexének meghatározása,
		\item A kép a szűrő ablakjának megfelelő (\texttt{nh\_N $\times$ nh\_N}) részének a globális (\texttt{im}) bufferből a lokális
		\texttt{Lwork} bufferbe való másolása,
		\item Lokális bufferben kiválasztásos részleges sorbarendezés. Részleges alatt a képrészlet feléig történő rendezést értem,
		\item Medián megállípítása és a kimeneti \texttt{oim} bufferbe írása.
	\end{enumerate*}

\cbstart
\noindent Továbbá a kimenet állítása során megvizsgálom, hogy a szűrő kimeneti értéke az eredetinél alacsonyabb vagy magasabb. Ha magasabb
	akkor esélyes, hogy az adott indexű pont környezetében nem volt részecske. Ennek megfelelően a szűrt értéket
	eldobom és az eredeti értéket változatlanul írom a kimeneti bufferbe. Ez a medián szűrő nemlineárisát csökkenti, hiszen a
	részecskéktől (pozitív kiugrásoktól) mentes területet alakhűbben (változás mentesen) engedi át.
\cbend


\section{Átlagolás}
	Az adaptív döntési szint számításához szükség van a differenciális kép átlagára és a szórására.
	A számítás során a képet az eszköz \texttt{CL\_DEVICE\_MAX\_COMPUTE\_UNITS} paraméterének megfelelően osztom fel részekre. A
	részösszegeket, így párhuzamosan tudom számítani.
	\noindent A kernel függvényének fejléce a következő:
\begin{lstlisting}[language=C++]
__kernel void average(	uint		N,		// 0
					uint		file_N,	// 1					
		__global		uchar	*Oim,	// 2 :: [N][file_N][file_N]
		__global		uint		*Oavg,	// 3 :: [N][file_N][lN]
		__local		uchar	*Lwork	// 4 :: [xN][yN]
)
\end{lstlisting}
	
\noindent A kernel program lépései a következők:
	\begin{enumerate*}
		\item A work-item globális és lokális indexének meghatározása,
		\item A képet a work-item-ek számának megfelelő részre bontom,
		\item A képrészletek a globális \texttt{Oim} bufferből a lokális \texttt{Lwork} bufferbe való másolása,
		\item Az átlag számításához ezek összegét számítom és a kimeneti \texttt{Oavg} bufferbe írom.
	\end{enumerate*}
	A részösszegekből a host oldalon történik az átlag számítása.

\section{Detektálás}
	\begin{enumerate*}
		\item A work-item globális és lokális indexének meghatározása,
		\item Az adaptív külszöb számítása és ennek megfelelő detektálás,
		\item Kiterjesztés és a flood-fill algoritmussal a ROI meghatározása:
		\begin{enumerate*}
			\item Megjelölt pixel keresése,
			\item A megjelölés adott számú környezetére való kiterjesztése,
			\item A kiterjesztés során a két legtávolabbi pont lesz a ROI (region of interest) határpontjai.
		\end{enumerate*}
		\item ROI-n belüli pontokból a részecske pozíciójának számítása momentum módszerrel,
		\item Eredmény mentése a globális memóriába.
	\end{enumerate*}
	
	