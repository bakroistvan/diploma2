% ----------------------------------------------------------------------------
\chapter{A kernel programok lépéseinek bemutatása}
% ----------------------------------------------------------------------------
A lokális memóriát két azonos nagyságú $im$ és $oim$ bufferre osztottam fel.

\section{Medián szűrés}
	\noindent A kernel program lépései a következők:
	\begin{enumerate*}
		\item A work-item globális és lokális indexének meghatározása,
		\item A kép egy részének a globális memóriából a lokális \texttt{Lwork} bufferbe való másolása,
		\item Lokális bufferben kiválasztásos sorbarendezés,
		\item Medián megállípítása és a kimeneti \texttt{oim} bufferbe írása.
	\end{enumerate*}

\section{Átlagolás}
	\begin{enumerate*}
		\item A work-item globális és lokális indexének meghatározása,
		\item A kép a work-item-ek számának megfelelő részre bontom,
		\item Az átlag számításához ezek összegét számítom és a kimeneti \texttt{oavg} bufferbe írom.
	\end{enumerate*}
	A részösszegekből a host oldalon történik az átlag számítása.

\section{Detektálás}
	\begin{enumerate*}
		\item A work-item globális és lokális indexének meghatározása,
		\item Kiterjesztés és a flood-fill algoritmussal a ROI meghatározása:
		\begin{enumerate*}
			\item Megjelölt pixel keresése,
			\item Adott környezetére való kiterjesztése,
			\item A kiterjesztés során a két legtávolabbi pont lesz a ROI határpontjai.
		\end{enumerate*}
		\item Részecske pozíciójának számítása momentum módszerrel,
		\item Eredmény mentése a globális memóriába.
	\end{enumerate*}

% \begin{lstlisting}[frame=single,float=!ht,caption=A detektálás kernelének kódja,
% label=listing:kernel]
% asd;
% \end{lstlisting}